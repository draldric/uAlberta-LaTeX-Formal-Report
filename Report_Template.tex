% Tips, Hints, and Useful Commands

%    \textsc{}          Small Caps
%    \textbf{}          Bold Text
%    \textit{}          Italic Text
%    \underline{}       Underline Text
%    \cite{<tag>}       Inserts a Citation
%    \Cref{<label>}     Inserts a Reference
%    \(<Stuff>\)        In-line Math
%    \vspace{xxcm}      Add Vertical Spacing in centimeters


\documentclass[12pt, letterpaper]{article}

\usepackage{lipsum}
\usepackage{amsmath}
\usepackage{amsfonts}
\usepackage{amssymb}
\usepackage{graphicx}
\usepackage[margin=1in]{geometry}
\usepackage{setspace}
\usepackage{tabularx}
\usepackage{multirow}
\usepackage{verbatim}
\usepackage{subcaption}
\usepackage[labelfont=bf]{caption}
\usepackage{hyperref}
\usepackage{cleveref}
\hypersetup{hidelinks,pdfencoding=unicode}
\usepackage[
	style=ieee,
	dashed=false,
	citestyle=numeric-comp,
	backend=bibtex,
	refsegment=section,
	sorting=none,
	defernumbers=true]{biblatex}

\newcolumntype{C}{>{\centering\arraybackslash}X}
\newcolumntype{L}{>{\raggedright\arraybackslash}X}
\newcolumntype{R}{>{\raggedleft\arraybackslash}X}
\renewcommand{\arraystretch}{1.5} % Changes the Vertical Padding in the Cells

\newcommand{\nonumeq}[1]{\begin{align*}#1\end{align*}}
\newcommand{\numeq}[2]{\begin{align}\label{#2}#1\end{align}}

% BIBLIOGRAPHY LOCATION
 % .  - This folder
 % .. - Up one Folder
 \addbibresource{./References/References.bib}

\begin{document}

%%%%%%%%%%%%%%%%%%%%%%%%%%%%%%%%%%%%%%%%%%%%%%%%%%%%%%%%%%%%%%%%%%%%%%%%%%%%%%%%
%%                                Title Page                                  %%
%%%%%%%%%%%%%%%%%%%%%%%%%%%%%%%%%%%%%%%%%%%%%%%%%%%%%%%%%%%%%%%%%%%%%%%%%%%%%%%%
	\begin{titlepage}
		\centering
		\includegraphics[width=0.2\textwidth]{Logo.jpg}\par\vspace{1cm}
	% Institution
		{\scshape\LARGE University of Alberta \par}
		\vspace{1cm}
	% Department
		{\scshape\Large Department of Mechanical Engineering\par}
		\vspace{1.5cm}
	% Title
		{\huge\bfseries Report Title\par}
		\vspace{2cm}
	% Author
		{\Large\itshape Last, First (Student Number)\par}
		\vfill
	% Additional Information
		\par % New Paragraph
		
		% Include Group Members
		\begin{comment} 
			\textbf{\underline{Group X}:}\par
			\vspace{0.5cm}
			\textsc{Last, First} (Student Number)\par
			\vspace{0.25cm}
			\textsc{Last, First} (Student Number)\par
			\vspace{0.25cm}
			\textsc{Last, First} (Student Number)\par
			\vspace{0.25cm}
			\textsc{Last, First} (Student Number)\par
			\vspace{0.25cm}
			\textsc{Last, First} (T.A.)\par
			\vspace{0.25cm}
		\end{comment}
		
		\vfill
	
	% Bottom of the page
			{\large \today\par} % For a Future Due Date Use {\large Month Day, Year\par} e.g. {\large February 23, 1993\par}
	\end{titlepage}
	
	\onehalfspacing % \doublespacing or \singlespacing 
	\setlength{\parskip}{1em} % Add Space Between Paragraphs
	\pagenumbering{roman} % Page Numbers set to Roman Numerals 
	\setcounter{page}{2} % Start at Page 2 to Count for the Title Page

%%%%%%%%%%%%%%%%%%%%%%%%%%%%%%%%%%%%%%%%%%%%%%%%%%%%%%%%%%%%%%%%%%%%%%%%%%%%%%%%
%%                                  Abstract                                  %%
%%%%%%%%%%%%%%%%%%%%%%%%%%%%%%%%%%%%%%%%%%%%%%%%%%%%%%%%%%%%%%%%%%%%%%%%%%%%%%%%
	\section*{Abstract}\hspace{2.6ex}\label{Abstract}
		\lipsum[23-25] % DELETE THIS LINE
		\newpage
	
	\tableofcontents
	\newpage
	
	\listoffigures
	\listoftables
	\newpage
        
%%%%%%%%%%%%%%%%%%%%%%%%%%%%%%%%%%%%%%%%%%%%%%%%%%%%%%%%%%%%%%%%%%%%%%%%%%%%%%%%
%%                                Nomenclature                                %%
%%%%%%%%%%%%%%%%%%%%%%%%%%%%%%%%%%%%%%%%%%%%%%%%%%%%%%%%%%%%%%%%%%%%%%%%%%%%%%%%
	\section*{Nomenclature}\label{}
		\begin{table*}[ht!]
			\label{}
			\begin{tabular}{r l}
				$Symbol 0$& Definition 0\\
				$Symbol 1$& Definition 1\\
				$Symbol 2$& Definition 2\\
				$Symbol n$& Definition n\\
				%$\#$& Universal Symbol for keyword, now used to `hashtag' selfies.\\
				%$DRA$& Initals of the original author of this template\\
			\end{tabular}
		\end{table*}
		\subsection*{Greek Letters}
			\begin{table*}[ht!]
				\label{}
				\begin{tabular}{r l}
					$Greek Symbol 0$& Definition 0\\  
					$Greek Symbol 1$& Definition 1\\  
					$Greek Symbol 2$& Definition 2\\
					$Greek Symbol n$& Definition n\\
					%$\pi$& Mathematical constant for the ratio of the circumference of a circle to the diameter.\\
					%$\tau$& Mathematical constant for the ratio of the circumference of a circle to the radius.\\
				\end{tabular}
			\end{table*}
	\newpage
        
%%%%%%%%%%%%%%%%%%%%%%%%%%%%%%%%%%%%%%%%%%%%%%%%%%%%%%%%%%%%%%%%%%%%%%%%%%%%%%%%
%%                               Introduction                                 %%
%%%%%%%%%%%%%%%%%%%%%%%%%%%%%%%%%%%%%%%%%%%%%%%%%%%%%%%%%%%%%%%%%%%%%%%%%%%%%%%%
	\pagenumbering{arabic} % Set Page numbers to Arabic Numbers and Start from 1
	\section{Introduction}\hspace{2.6ex}
			\label{}
			\lipsum[29-32] % DELETE THIS LINE
        
%%%%%%%%%%%%%%%%%%%%%%%%%%%%%%%%%%%%%%%%%%%%%%%%%%%%%%%%%%%%%%%%%%%%%%%%%%%%%%%%
%%                          Equipment And Procedure                           %%
%%%%%%%%%%%%%%%%%%%%%%%%%%%%%%%%%%%%%%%%%%%%%%%%%%%%%%%%%%%%%%%%%%%%%%%%%%%%%%%%
	\section{Equipment \& Procedure}\label{}
		\subsection{Equipment}\hspace{2.6ex}
			\lipsum[4-7] % DELETE THIS LINE
			\begin{itemize}
				\item
				{
					\textsc{Equipment 1}
					\begin{itemize}
						\item Info 1
						\item Info 2
						\item Info 3
					\end{itemize}
				}
				\item
				{
					\textsc{Equipment 2}
					\begin{itemize}
						\item Info 1
						\item Info 2
						\item Info 3
					\end{itemize}
				}
			\end{itemize}
		\subsection{Procedure}\hspace{2.6ex}
			\lipsum[13-15] % DELETE THIS LINE
			\begin{enumerate}
				\item
				{
					\textsc{Step 1}
					\begin{itemize}
						\item Info 1
					\end{itemize}
				}
				\item
				{
					\textsc{Step 2}
					\begin{itemize}
						\item Info 1
						\item Info 2
					\end{itemize}
				}
				\item
				{
					\textsc{Step 3}
				}
				\item
				{
					\textsc{Step 4}
					\begin{itemize}
						\item Info 1
					\end{itemize}
				}
			\end{enumerate}
            
%%%%%%%%%%%%%%%%%%%%%%%%%%%%%%%%%%%%%%%%%%%%%%%%%%%%%%%%%%%%%%%%%%%%%%%%%%%%%%%%
%%                              Theory Section                                %%
%%%%%%%%%%%%%%%%%%%%%%%%%%%%%%%%%%%%%%%%%%%%%%%%%%%%%%%%%%%%%%%%%%%%%%%%%%%%%%%%
	\section{Theory}\hspace{2.6ex}
		\label{}
		\lipsum[3] % DELETE THIS LINE
		
		\begin{align} % Creates Numbered Equation 
			\label{eq:eg1} E = & m\times c^2
		\end{align}
		\newline Where, \newline
		\indent \(E\) := is the Energy,\newline
		\indent \(m\) := is the Mass, and \newline
		\indent \(c\) := is the Speed of Light in a Vacuum.

	\begin{align*} % Creates Unnumbered Equation 
		KE = & \cfrac{1}{2}m\times v^2
	\end{align*}
	
	\begin{align} % Creates Numbered and Aligned Equations 
		\label{eq:eg3} E = & m\times c^2\\
		\label{eq:eg4} E^2 = & m^2\times c^4 + \rho^2\times c^2
	\end{align}

%%%%%%%%%%%%%%%%%%%%%%%%%%%%%%%%%%%%%%%%%%%%%%%%%%%%%%%%%%%%%%%%%%%%%%%%%%%%%%%%
%%                          Results And Discussions                           %%
%%%%%%%%%%%%%%%%%%%%%%%%%%%%%%%%%%%%%%%%%%%%%%%%%%%%%%%%%%%%%%%%%%%%%%%%%%%%%%%%
	\section{Results \& Discussion}\hspace{2.6ex}
			\label{}
			\lipsum[45-57] % DELETE THIS LINE
	\newpage
	
%%%%%%%%%%%%%%%%%%%%%%%%%%%%%%%%%%%%%%%%%%%%%%%%%%%%%%%%%%%%%%%%%%%%%%%%%%%%%%%%
%%                         EXAMPLE SECTION (REMOVE)                           %%
%%%%%%%%%%%%%%%%%%%%%%%%%%%%%%%%%%%%%%%%%%%%%%%%%%%%%%%%%%%%%%%%%%%%%%%%%%%%%%%%
	\section{EXAMPLE SECTION PLEASE REMOVE}
		This section aims to provide examples on how to structure and create specific components in your report document. The very first one is showing a citation, like the one at the end of this sentence \cite{TEST}. The second shows how to create more than one citation and how they are grouped \cite{testone,cite2,cite3,cite4,cite5}.
		This sentence shows how a gap in the citations is handled \cite{testone,cite2,cite3,cite5}. 
		\subsection{Tables}

			% L - Left Aligned (Equal Spacing)
			% C - Center Aligned (Equal Spacing)
			% R - Right Aligned (Equal Spacing)
			% l - Left Aligned (Fit to Contents)
			% c - Center Aligned (Fit to Contents)
			% r - Right Aligned (Fit to Contents)

			\begin{table}[!htb]
				\caption{This is a basic table}
				\centering
				\begin{tabularx}{0.75\textwidth}{LCR} 
					% Equally spaced cells that are left, center, and reight aligned. 
					% The entire table will be 75% the width of the text.
					\hline
					\textbf{Left Aligned Title} & \textbf{Centered Title} & \textbf{Right Aligned Title} \\\hline
					This is left aligned & This is centered & This is right aligned \\
					This is left aligned & This is centered & This is right aligned \\
					This is left aligned & This is centered & This is right aligned \\
					This is left aligned & This is centered & This is right aligned \\\hline
				\end{tabularx}
				\label{tab:basicTable}
			\end{table}

			\begin{table}[!htb]
				\caption{This is a complex table.}
				\centering
				\begin{tabularx}{\textwidth}{lCR}
					% Left most cell is fitted to the content.
					% The center and right columns are equally spaced cells that are center, and reight aligned. 
					% The entire table will be 75% the width of the text.
					\hline
					\multirow{2}{*}{\textbf{This is two row\quad}} & \multicolumn{2}{c}{\textbf{This is two columns}}\\\cline{2-3} % \cline draws a partial line across cells #-#
					& \textbf{Centered Title} & \textbf{Right Aligned Title} \\\hline
					\multirow{2}{*}{This is two row} & This is centered & This is right aligned \\
					& This is centered & This is right aligned \\\cline{1-1}
					\multirow{2}{*}{This is two row} & This is centered & This is right aligned \\
					& This is centered & This is right aligned \\\hline
				\end{tabularx}
				\label{tab:complexTable}
			\end{table}


		\subsection{Figures}
			This section will provide examples of how to create figures, and different types of multi/sub-figures. Additionally, if you have many figures in a section and they are bleeding too much into the following sections a \textbackslash{}clearpage command can be issued before the next section. However, note that this will force the next section to begin on a new page. 
			\begin{figure}[!htb]
				\centering
				\includegraphics[width=0.7\textwidth]{example-image}
				\caption{This is an example of a single image figure.}
				\label{fig:singleImage}
			\end{figure}

			\begin{figure}[!htb]
				\centering
				\begin{subfigure}{0.45\textwidth}
					\includegraphics[width=\textwidth]{example-image}
					\caption{} % Leave blank for just letter
					\label{fig:doubleImage:a}
				\end{subfigure}
				~
				\begin{subfigure}{0.45\textwidth}
					\includegraphics[width=\textwidth]{example-image}
					\caption{} % Leave blank for just letter
					\label{fig:doubleImage:b}
				\end{subfigure}
				\caption{This is an example of a double image figure.}
				\label{fig:doubleImage}
			\end{figure}

			\begin{figure}[!htb]
				\centering
				\hspace*{\fill}% Adds space to left of top image (prevents two images from going to top)
				\begin{subfigure}{0.45\textwidth}
					\includegraphics[width=\textwidth]{example-image}
					\caption{} % Leave blank for just letter
					\label{fig:tripleImage:a}
				\end{subfigure}
				\hspace*{\fill} % Adds space to right of top image (prevents two images from going to top)
				\par\vspace{1em}% Adds space between upper and lower images
				\begin{subfigure}{0.45\textwidth}
					\includegraphics[width=\textwidth]{example-image}
					\caption{} % Leave blank for just letter
					\label{fig:tripleImage:b}
				\end{subfigure}
				~ % Adds space between the two lower figures
				\begin{subfigure}{0.45\textwidth}
					\includegraphics[width=\textwidth]{example-image}
					\caption{} % Leave blank for just letter
					\label{fig:tripleImage:c}
				\end{subfigure}
				\caption{This is an example of a triple image figure.}
				\label{fig:tripleImage}
			\end{figure}

			\begin{figure}[!htb]
				\centering
				\hspace*{\fill}% Adds space to left of top image (prevents two images from going to top)
				\begin{subfigure}{\dimexpr 0.90\textwidth+1em\relax} % 0.9 = 0.45 + 0.45, and 1em is the width of ~
					\includegraphics[width=\textwidth]{example-image}
					\caption{} % Leave blank for just letter
					\label{fig:tripleImage:a}
				\end{subfigure}
				\hspace*{\fill} % Adds space to right of top image (prevents two images from going to top)
				\par\vspace{1em}% Adds space between upper and lower images
				\begin{subfigure}{0.45\textwidth}
					\includegraphics[width=\textwidth]{example-image}
					\caption{} % Leave blank for just letter
					\label{fig:tripleImage:b}
				\end{subfigure}
				~ % Adds space between the two lower figures
				\begin{subfigure}{0.45\textwidth}
					\includegraphics[width=\textwidth]{example-image}
					\caption{} % Leave blank for just letter
					\label{fig:tripleImage:c}
				\end{subfigure}
				\caption{This is a second example of a triple image figure.}
				\label{fig:tripleImage}
			\end{figure}

			\begin{figure}[!htb]
				\centering
				\begin{subfigure}{0.45\textwidth}
					\includegraphics[width=\textwidth]{example-image}
					\caption{} % Leave blank for just letter
					\label{fig:quadImage:a}
				\end{subfigure}
				~ % Adds space between the two top figures
				\begin{subfigure}{0.45\textwidth}
					\includegraphics[width=\textwidth]{example-image}
					\caption{} % Leave blank for just letter
					\label{fig:quadImage:b}
				\end{subfigure}
				\par\vspace{1em} % Adds space between upper and lower images
				\begin{subfigure}{0.45\textwidth}
					\includegraphics[width=\textwidth]{example-image}
					\caption{} % Leave blank for just letter
					\label{fig:quadImage:c}
				\end{subfigure}
				~ % Adds space between the two lower figures
				\begin{subfigure}{0.45\textwidth}
					\includegraphics[width=\textwidth]{example-image}
					\caption{} % Leave blank for just letter
					\label{fig:quadImage:d}
				\end{subfigure}
				\caption{This is an example of a quad image figure.}
				\label{fig:quadImage}
			\end{figure}

			\clearpage % forces the remaining images (floats to be placed)
		
		
		\subsection{Equations}
			The following equation has no referencing number:
			\nonumeq{E = & m\ c^2}

			\Cref{eq:quickEq} has a reference to it though. Or for more control the source for \Cref{eq:quickEq} can be written out fully as it was for \Cref{eq:quickEq2}.

			\numeq{pi = & 3.1415...}{eq:quickEq} % shorthand for the following way of writing equations.
			\begin{align}\label{eq:quickEq2}
				e = & 2.7183...
			\end{align}

			If you have multiple equations that you want arranged very neatly, use the align environment and you can assign individual equations numbers as shown in \Cref{eq:multiref:a,eq:multiref:b,eq:multiref:c}.
			\begin{align}%Note: Alignment happens at the "=" character
				\label{eq:multiref:a} Equation1 = & 1 + 1\\
				\label{eq:multiref:b} Equation2 = & 2 + 2\\
				\label{eq:multiref:c} Equation3 = & 3 + 3
			\end{align}
		
%%%%%%%%%%%%%%%%%%%%%%%%%%%%%%%%%%%%%%%%%%%%%%%%%%%%%%%%%%%%%%%%%%%%%%%%%%%%%%%%
%%                                Conclusion                                  %%
%%%%%%%%%%%%%%%%%%%%%%%%%%%%%%%%%%%%%%%%%%%%%%%%%%%%%%%%%%%%%%%%%%%%%%%%%%%%%%%%
	\section{Conclusion}\hspace{2.6ex}
		\label{}
		\lipsum[87-89] % DELETE THIS LINE
	
%%%%%%%%%%%%%%%%%%%%%%%%%%%%%%%%%%%%%%%%%%%%%%%%%%%%%%%%%%%%%%%%%%%%%%%%%%%%%%%%
%%                          Technical Recommendation                          %%
%%                                 (Optional)                                 %%
%%%%%%%%%%%%%%%%%%%%%%%%%%%%%%%%%%%%%%%%%%%%%%%%%%%%%%%%%%%%%%%%%%%%%%%%%%%%%%%%
	\section{Technical Recommendations}\hspace{2.6ex}
		\label{}
		\lipsum[100] % DELETE THIS LINE
	\newpage
        
%%%%%%%%%%%%%%%%%%%%%%%%%%%%%%%%%%%%%%%%%%%%%%%%%%%%%%%%%%%%%%%%%%%%%%%%%%%%%%%%
%%                             Figures And Tables                             %%
%%                                 (Optional)                                 %%
%%%%%%%%%%%%%%%%%%%%%%%%%%%%%%%%%%%%%%%%%%%%%%%%%%%%%%%%%%%%%%%%%%%%%%%%%%%%%%%%
	\section{Figures \& Tables}
		\subsection{Figures}
			\begin{figure}[ht!]
				\centering
				\includegraphics[width= 0.75\columnwidth]{example-image}
				\caption{\textbf{Figure Caption 0}}
				\label{fig:example1}
			\end{figure}
			\vspace{0.5cm}
			\clearpage
			\begin{figure}[ht!]
				\centering
				\includegraphics[width= 0.75\columnwidth]{example-image}
				\caption{Figure Caption 1}
				\label{fig:example2}
			\end{figure}
			\vspace{0.5cm}
			
			\begin{figure}[ht!]
				\centering
				\includegraphics[width= 0.75\columnwidth]{example-image}
				\caption{Figure Caption n}
				\label{fig:example3}
			\end{figure}
			\vspace{0.5cm}
			\clearpage
		\subsection{Tables}
			\begin{table}[ht!]
				\centering
				\caption{Table Caption 0}\label{tab:example1}
				\begin{tabularx}{\textwidth}{|LR C|C|}
					\hline % Top Line
					\multicolumn{2}{|c|}{\multirow{2}{*}{Cell 1}} & Cell 2 & Cell 3\\
					\multicolumn{2}{|c|}{} & Cell 4 & Cell 5\\ \hline
					Cell 6 & Cell 7 & Cell 8 & Cell 9\\
					\hline % Bottom Line
				\end{tabularx}
			\end{table}
			\vspace{0.5cm}
			
			\begin{table}[ht!]
				\centering
				\caption{Table Caption n}\label{tab:example2}
				\begin{tabularx}{\textwidth}{|L|C|C|R|}
					\hline % Top Line
					Cell 1 & Cell 2 & Cell 3 & Cell 4\\
					Cell 5 & Cell 6 & Cell 7 & Cell 8\\
					Cell 9 & Cell 10 & Cell 11 & Cell 12\\
					\hline % Bottom Line
				\end{tabularx}
			\end{table}
			\vspace{0.5cm}
                
%%%%%%%%%%%%%%%%%%%%%%%%%%%%%%%%%%%%%%%%%%%%%%%%%%%%%%%%%%%%%%%%%%%%%%%%%%%%%%%%
%%                                References                                  %%
%%%%%%%%%%%%%%%%%%%%%%%%%%%%%%%%%%%%%%%%%%%%%%%%%%%%%%%%%%%%%%%%%%%%%%%%%%%%%%%%
	\clearpage
	\printbibliography[heading=bibintoc]
	\newpage
        
%%%%%%%%%%%%%%%%%%%%%%%%%%%%%%%%%%%%%%%%%%%%%%%%%%%%%%%%%%%%%%%%%%%%%%%%%%%%%%%%
%%                                Appendices                                  %%
%%%%%%%%%%%%%%%%%%%%%%%%%%%%%%%%%%%%%%%%%%%%%%%%%%%%%%%%%%%%%%%%%%%%%%%%%%%%%%%%
	\begin{appendix}
	%%%%% Appendix A %%%%%
		\pagenumbering{arabic}
		\renewcommand*{\thepage}{A\arabic{page}}
		
		\section{Sample Calculations}
		\{Insert Text Here\}
		
		\subsection{Calculations}
		\{Insert Text Here\}
		\newpage
		
		\subsection{Uncertainty Analysis}
		\{Insert Text Here\}
		\newpage
	%%%%% Appendix B %%%%%
		\pagenumbering{arabic}
		\renewcommand*{\thepage}{B\arabic{page}}
		
		\section{Raw Data}
		\{Insert Text Here\}
		\newpage
	%%%%% Appendix C %%%%%
		%\pagenumbering{arabic}
		%\renewcommand*{\thepage}{C\arabic{page}}
		%
		%\section{Section Title for Appendix C}
		%\{Insert Text Here\}
		%\newpage
	\end{appendix}
\end{document}
